% Use    $ pdflatex Misaka.tex
\documentclass[conference]{IEEEtran}
\IEEEoverridecommandlockouts
% The preceding line is only needed to identify funding in the first footnote. If that is unneeded, please comment it out.
\usepackage{cite}
\usepackage{amsmath,amssymb,amsfonts}
\usepackage{algorithmic}
\usepackage{graphicx}
\usepackage{textcomp}
\usepackage{xcolor}
\def\BibTeX{{\rm B\kern-.05em{\sc i\kern-.025em b}\kern-.08em
    T\kern-.1667em\lower.7ex\hbox{E}\kern-.125emX}}
\begin{document}

\title{Moveable Interactive Swarm Platform for Smart Grid Technology Development Test and Evaluation\\}

\author{\IEEEauthorblockN{1\textsuperscript{st} Tingliang Zhang}
\IEEEauthorblockA{\textit{Dept. Electrical Engineering} \\
\textit{Tsinghua University}\\
Beijing, China \\
zhangtl16@mails.tsinghua.edu.cn}
\and
\IEEEauthorblockN{2\textsuperscript{nd} Haiwang Zhong}
\IEEEauthorblockA{\textit{Dept. Electrical Engineering} \\
\textit{Tsinghua University}\\
Beijing, China \\
zhonghw@tsinghua.edu.cn}
% \and
% \IEEEauthorblockN{3\textsuperscript{rd} Given Name Surname}
% \IEEEauthorblockA{\textit{dept. name of organization (of Aff.)} \\
% \textit{name of organization (of Aff.)}\\
% City, Country \\
% email address or ORCID}
% \and
% \IEEEauthorblockN{4\textsuperscript{th} Given Name Surname}
% \IEEEauthorblockA{\textit{dept. name of organization (of Aff.)} \\
% \textit{name of organization (of Aff.)}\\
% City, Country \\
% email address or ORCID}
% \and
% \IEEEauthorblockN{5\textsuperscript{th} Given Name Surname}
% \IEEEauthorblockA{\textit{dept. name of organization (of Aff.)} \\
% \textit{name of organization (of Aff.)}\\
% City, Country \\
% email address or ORCID}
% \and
% \IEEEauthorblockN{6\textsuperscript{th} Given Name Surname}
% \IEEEauthorblockA{\textit{dept. name of organization (of Aff.)} \\
% \textit{name of organization (of Aff.)}\\
% City, Country \\
% email address or ORCID}
}

\maketitle

\begin{abstract}
    In this article, we present Misaka, a visualized swarm testbed for smart grid algorithm evaluation, also an extendable open-source open-hardware platform for developing tabletop tangible swarm interfaces.

    The platform consists of a collection of custom-designed 3-onmi-wheeled robots each 10 cm in diameter, high accuracy localization through a microdot pattern overlaid on top of the activity sheets, and a software framework for application development and control, while remaining affordable (per unit cost about 30 USD at the prototype stage). We illustrate the potential of tabletop swarm user interfaces through a set of smart grid algorithm development application scenarios developed with Misaka.
\end{abstract}

\begin{IEEEkeywords}
Smart grid, Human-Robot Interaction, Swarm user interfaces, Tangible Robots, Testbed
\end{IEEEkeywords}

\section{Introduction}


To stimulate future research on swarm user interfaces, we distribute our Misaka tabletop swarm user interface platform in open-source and open-hardware.



In summary, our contributions are:

\begin{itemize}
    \item The first open-source hardware/software platform for smart grid algorithm evaluation with tabletop swarm user interfaces
    \item Redefinition for Algorithm Visualization with several implemented examples
    \item A set of scenarios to illustrate the unprecedented possibilities offered by Misaka swarm and by tabletop swarm user interfaces in general
    \item A common platform for any algorithm visualization, and other interactive swarm user interfaces
\end{itemize}

Furthermore, as benefits, Misaka:

\begin{itemize}
    \item are modular each unit, can be extended with powerful platform, such as NVIDIA Jetson NANO.
    \item can simulate real smart grid communication scenarios with Zigbee modular on board
    \item can interact with user moving and turning the robot or using the illuminated capacitive touch keys on the robot 
    \item can be manipulated either individually or collectively
    \item are small enough to also act as “pixels” of a physical display
    \item can coexist in large numbers
    \item are lightweight, can operate on any horizontal surface, and relatively cost-effective: about 30 USD each now, down to \$15 if mass manufactured.
\end{itemize}

\section*{Acknowledgment}

The preferred spelling of the word ``acknowledgment'' in America is without 
an ``e'' after the ``g''. Avoid the stilted expression ``one of us (R. B. 
G.) thanks $\ldots$''. Instead, try ``R. B. G. thanks$\ldots$''. Put sponsor 
acknowledgments in the unnumbered footnote on the first page.

\section*{References}

Please number citations consecutively within brackets \cite{b1}. The 
sentence punctuation follows the bracket \cite{b2}. Refer simply to the reference 
number, as in \cite{b3}---do not use ``Ref. \cite{b3}'' or ``reference \cite{b3}'' except at 
the beginning of a sentence: ``Reference \cite{b3} was the first $\ldots$''

Number footnotes separately in superscripts. Place the actual footnote at 
the bottom of the column in which it was cited. Do not put footnotes in the 
abstract or reference list. Use letters for table footnotes.

Unless there are six authors or more give all authors' names; do not use 
``et al.''. Papers that have not been published, even if they have been 
submitted for publication, should be cited as ``unpublished'' \cite{b4}. Papers 
that have been accepted for publication should be cited as ``in press'' \cite{b5}. 
Capitalize only the first word in a paper title, except for proper nouns and 
element symbols.

For papers published in translation journals, please give the English 
citation first, followed by the original foreign-language citation \cite{b6}.

\begin{thebibliography}{00}
\bibitem{b1} G. Eason, B. Noble, and I. N. Sneddon, ``On certain integrals of Lipschitz-Hankel type involving products of Bessel functions,'' Phil. Trans. Roy. Soc. London, vol. A247, pp. 529--551, April 1955.
\bibitem{b2} J. Clerk Maxwell, A Treatise on Electricity and Magnetism, 3rd ed., vol. 2. Oxford: Clarendon, 1892, pp.68--73.
\bibitem{b3} I. S. Jacobs and C. P. Bean, ``Fine particles, thin films and exchange anisotropy,'' in Magnetism, vol. III, G. T. Rado and H. Suhl, Eds. New York: Academic, 1963, pp. 271--350.
\bibitem{b4} K. Elissa, ``Title of paper if known,'' unpublished.
\bibitem{b5} R. Nicole, ``Title of paper with only first word capitalized,'' J. Name Stand. Abbrev., in press.
\bibitem{b6} Y. Yorozu, M. Hirano, K. Oka, and Y. Tagawa, ``Electron spectroscopy studies on magneto-optical media and plastic substrate interface,'' IEEE Transl. J. Magn. Japan, vol. 2, pp. 740--741, August 1987 [Digests 9th Annual Conf. Magnetics Japan, p. 301, 1982].
\bibitem{b7} M. Young, The Technical Writer's Handbook. Mill Valley, CA: University Science, 1989.
\end{thebibliography}

\end{document}
