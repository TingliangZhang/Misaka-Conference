% Use    $ latexmk -bibtex -pdf
\documentclass[conference]{IEEEtran}
\IEEEoverridecommandlockouts
% The preceding line is only needed to identify funding in the first footnote. If that is unneeded, please comment it out.
\usepackage{cite}
\usepackage{amsmath,amssymb,amsfonts}
\usepackage{algorithmic}
\usepackage{graphicx}
\usepackage{textcomp}
\usepackage{xcolor}
\def\BibTeX{{\rm B\kern-.05em{\sc i\kern-.025em b}\kern-.08em
    T\kern-.1667em\lower.7ex\hbox{E}\kern-.125emX}}
\begin{document}

\title{Moveable Interactive Swarm Platform for Smart Grid Technology Development Test and Evaluation\\}

\author{\IEEEauthorblockN{1\textsuperscript{st} Tingliang Zhang}
\IEEEauthorblockA{\textit{Dept. Electrical Engineering} \\
\textit{Tsinghua University}\\
Beijing, China \\
zhangtl16@mails.tsinghua.edu.cn}
\and
\IEEEauthorblockN{2\textsuperscript{nd} Haiwang Zhong}
\IEEEauthorblockA{\textit{Dept. Electrical Engineering} \\
\textit{Tsinghua University}\\
Beijing, China \\
zhonghw@tsinghua.edu.cn}
}

\maketitle

\begin{abstract}
    In this article, we present Misaka, a visualized swarm testbed for smart grid algorithm evaluation, also an extendable open-source open-hardware platform for developing tabletop tangible swarm interfaces\cite{yan2019consensus}.

    The platform consists of a collection of custom-designed 3-onmi-wheeled robots each 10 cm in diameter, high accuracy localization through a microdot pattern overlaid on top of the activity sheets, and a software framework for application development and control, while remaining affordable (per unit cost about 30 USD at the prototype stage). We illustrate the potential of tabletop swarm user interfaces through a set of smart grid algorithm development application scenarios developed with Misaka.
\end{abstract}

\begin{IEEEkeywords}
    Smart grid, Human-Robot Interaction, Swarm user interfaces, Tangible Robots, Testbed
\end{IEEEkeywords}

\section{Introduction}

Most recently, an increasing penetration of distributed energy resources (DERs) in Energy Internet imposes great challenges to conventional centralized economic dispatch\cite{yan2019consensus}.

In the situation where many nodes need to be optimized, the traditional centralized optimization has problems such as communication congestion, node failure, and insufficient computing power.

There are many tangible robots such as Zooids\cite{le2016zooids} and Cellulo\cite{ozgur2017cellulo}.

To stimulate future research on swarm user interfaces, we distribute our Misaka tabletop swarm user interface platform in open-source and open-hardware.



In summary, our contributions are:

\begin{itemize}
    \item The first open-source hardware/software platform for smart grid algorithm evaluation with tabletop swarm user interfaces
    \item Redefinition for Algorithm Visualization with several implemented examples
    \item A set of scenarios to illustrate the unprecedented possibilities offered by Misaka swarm and by tabletop swarm user interfaces in general
    \item A common platform for any algorithm visualization, and other interactive swarm user interfaces
\end{itemize}

Furthermore, as benefits, Misaka:

\begin{itemize}
    \item are modular each unit, can be extended with powerful platform, such as NVIDIA Jetson NANO.
    \item can simulate real smart grid communication scenarios with Zigbee modular on board
    \item can interact with user moving and turning the robot or using the illuminated capacitive touch keys on the robot 
    \item can be manipulated either individually or collectively
    \item are small enough to also act as “pixels” of a physical display
    \item can coexist in large numbers
    \item are lightweight, can operate on any horizontal surface, and relatively cost-effective: about 30 USD each now, down to \$15 if mass manufactured.
\end{itemize}



\bibliographystyle{IEEEtran}
\bibliography{IEEEabrv,refs}

\end{document}
